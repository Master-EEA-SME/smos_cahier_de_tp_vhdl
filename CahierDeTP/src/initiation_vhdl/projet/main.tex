\section{Mini projet - Horloge}

L'horloge n'affichera que les \textbf{heures} et les \textbf{minutes}. L'affichage des dizaine et unités des heures et des minutes se fera sur afficheur sept segments chacun.

\medskip

L'horloge aura deux modes de fonctionnement. Un mode normal où l'heure compte normalement et un mode configuration où on pourra venir modifier les heures et les minutes.

\medskip
Si on appuie sur \textit{KEY1}, on rentre en mode configuration. L'horloge arrête de compter tant qu'on est dans le mode configuration.

\medskip

Dans le mode configuration lorsqu'on appuie sur \textit{KEY0} ou \textit{KEY2}, on incrémentera ou décrémentera respectivement soit les heures soit les minutes.

\medskip

Lorsqu'on entre en mode configuration, ce sont les minutes qui pourront être modifiées. Lors d'un nouvel appui sur \textit{KEY1} et ce seront les heures qui pourront être modifiées. Un nouvel appui sur \textit{KEY1}, on revient en mode normal.

\medskip

- En utilisant les composants et en modélisant des nouveaux, faire un schéma fonctionnel et les machines à états nécessaires pour modéliser l'horloge.

\medskip

- Implémenter votre modélisation en VHDL.

\medskip

- Valider votre implémentation en simulation.

\medskip

- Valider votre implémentation sur votre carte.
