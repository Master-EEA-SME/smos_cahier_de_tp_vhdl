\subsection{Compteur}
\label{sec:BasicCnt}
Voici la table de vérité d'un simple compteur : 

\begin{table}[ht]
    \centering
    \begin{tabular}{c c c c|c} 
        $ARst\_N$ & $Clk$ & $SRst$ &$En$ & $Q_t$ \\ 
        \hline
        0 & *           & * & * & 0 \\
        1 & \risingedge & 1 & * & 0 \\
        1 & \risingedge & 0 & 0 & $Q_{t-1}$ \\
        1 & \risingedge & 0 & 1 & $Q_{t-1} + 1$

    \end{tabular}
    \caption{Table de vérité d'un compteur}
    \label{ttab:BasicCnt}
\end{table}

Implémenter cette table de vérité en VHDL en utilisant un \textit{process} et \textbf{respectant bien la priorité}. Contraigner le vecteur $Q_n$ sur \textbf{8 bits}.

\medskip

Simuler votre compteur et vérifier que votre implémentation respecte bien la table de vérité.


\subsection{Cascade de compteurs}

Faire un compteur qui comptera toutes les secondes. Afficher la valeur de ce compteur sur un afficheur sept segments. Ce compteur devra compter 0 à 15.

\medskip

Faire un schéma fonctionnel en faisant apparaître deux compteurs en cascades. \\ 
\textbf{NB: Il est interdit de connecter aux entrées horloges autres que des signaux horloges}.

\medskip

Implémenter votre schéma fonctionnel en VHDL.

\medskip

A l'aide de l'analyseur logique (Signal Tap Logic Analyzer) visualiser le signal \textit{Q} des compteurs ainsi que le signal \textit{En} du second compteur. 

\medskip

Compiler, programmer et en utilisant l'analyseur logique, déclencher l'acquisition lorsque \textit{En} vaut $1$. Vérifier que \textit{En} est à $1$ que pendant un seul coup d'horloge.

