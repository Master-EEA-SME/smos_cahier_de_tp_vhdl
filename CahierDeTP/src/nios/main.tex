\chapter{Utilisation du processeur Nios}
\section{Introduction}
Le but de ce TP est de prendre en main l'outil \textit{Platform Designer} afin de pouvoir mettre en œuvre des solutions complexes et ciblés basé sur le processeur \textit{Nios II} et le bus \textit{Avalon}.


\section{Construction d'un microcontrôleur personnalisé}
\label{sec:BuildMCU}
A l'aide de l'outil \textit{Platform Designer}, construire un microcontrôleur basé sur le processeur \textit{Nios II} en ajoutant les périphériques suivants :
\begin{itemize}
    \item Nios II
    \item On Chip RAM (où le code sera stocké)
    \item JTAG UART (pour avoir une console série + pouvoir debugger)
    \item PIO (pour pouvoir contrôler des Leds)
\end{itemize}

\medskip

Générer le VHDL.

\medskip

Connecter la sortie du périphérique \textit{PIO} aux Leds de la carte DE0 Nano.

\medskip

Compiler et téléverser la configuration du FPGA vers votre carte.

\medskip

En utilisant le logiciel \textit{Nios II Software Build Tools for Eclipse} créer un projet en prenant comme modèle de projet \textbf{Hello World Small}.

\medskip

Sans toucher au code qui a été généré, compiler et téléverser vers le Nios. Vérifier que \textit{"Hello from Nios II!"} s'affiche bien sur la console \textit{Nios II Console}.

\medskip

Écrire un code qui permet de faire un chenillard va et vient sur les Leds en utilisant le périphérique PIO.

\section{Ajout d'un périphérique personnalisé au microcontrôleur}
\subsection{PWM}

On veut créer un composant VHDL qui permet de générer un signal PWM d'haute résolution sur une broche. Ce composant on veut l'interfacer avec le processeur NIOS II à l'aide du bus Avalon. Le PWM aura une résolution de \textbf{32 bits}.

\medskip

- Faire un diagramme fonctionnel du composant PWM sans prendre en compte le NIOS ni le bus Avalon.

\medskip

- Décrire l'entité et implémenter l'architecture en utilisant le diagramme fonctionnel.

\medskip

- Simuler et vérifier que votre composant crée bien un signal PWM avec la fréquence et le rapport cyclique variable.

\subsection{Interface avec le bus Avalon}

- Créer un composant VHDL qui permet d'interfacer le composant PWM que vous venez créer, avec le processeur Nios II à l'aide du bus Avalon. La table d'adressage doit être la suivante :

\begin{table}[ht]
    \centering
    \begin{bytefield}[endianness=big, bitwidth=0.5cm, bitheight=0.5cm, rightcurly=., rightcurlyspace=0pt]{32}
        \bitheader{0,7,8,15,16,23,24,31} \\
        \begin{rightwordgroup}{00h}
            \bitbox{32}{Duty}
        \end{rightwordgroup} \\
        \begin{rightwordgroup}{01h}
            \bitbox{32}{Freq}
        \end{rightwordgroup}
    \end{bytefield}
    \caption{Table d'adressage du composant PWM sur le bus Avalon}
\end{table}

- En utilisant l'outil \textit{Platform Designer}, ajouter votre composant d'interface que vous venez de créer pour qu'il soit reconnu comme un périphérique Avalon.

\medskip

- Intégrer ce composant au microcontrôleur que vous avez créer dans l’exercice \ref{sec:BuildMCU} et exporter la sortie du composant PWM. Générer le VHDL du système.

\medskip

- Affecter la sortie du PWM à une broche quelconque.

\medskip

- Compiler et téléverser la nouvelle configuration du FPGA vers votre carte.

\medskip

- Avec \textit{Nios II Software Build Tools for Eclipse}, tester votre périphérique en changeant le rapport cyclique et la fréquence. Valider le fonctionnent à l'aide d'un oscilloscope.