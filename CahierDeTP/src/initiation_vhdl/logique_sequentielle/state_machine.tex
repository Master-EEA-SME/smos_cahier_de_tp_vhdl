\subsection{Machine à états}
On veut modéliser un composant qui va commander un compteur.

Lorsqu'on appuie sur le bouton incrémenter, cela incrémentera de $1$ le compteur.

Lorsqu'on appuie sur le bouton décrémenter, cela décrémentera de $1$ le compteur.

Si les deux boutons sont appuyés, rien ne se passe.

Le contenu du compteur sera visualisé sur un afficheur sept-segments.

\medskip

- Quelle fonctionnalité faut-il ajouter au compteur de \ref{sec:BasicCnt} pour pouvoir répondre au besoin ci-dessus. Compléter la table de vérité \ref{ttab:BasicCnt}. Implémenter cette fonctionnalité. Simuler le compteur.


\medskip

- Modéliser la machine à états qui implémente la fonctionnalité d'un bouton. Créer le composant \textit{fsm\_btn} et implémenter la machine à états que vous avez modélisé. Simuler le composant.

\medskip

- Faire un schéma fonctionnel du problème en incluant le composant \textit{fsm\_btn} et qui intègre les deux boutons. Implémenter votre schéma fonctionnel en VHDL. Simuler.

\medskip

- Compiler et valider le bon fonctionnement sur la carte.
